% Created 2013-12-19 Thu 18:18
\documentclass[11pt]{article}
\usepackage[utf8]{inputenc}
\usepackage[T1]{fontenc}
\usepackage{fixltx2e}
\usepackage{graphicx}
\usepackage{longtable}
\usepackage{float}
\usepackage{wrapfig}
\usepackage{soul}
\usepackage{textcomp}
\usepackage{marvosym}
\usepackage{wasysym}
\usepackage{latexsym}
\usepackage{amssymb}
\usepackage{hyperref}
\tolerance=1000
\usepackage{hyperref}
\usepackage{amsmath}
\usepackage{caption}
\usepackage{subcaption}
\usepackage{graphicx}
\usepackage[usenames,dvipsnames,svgnames,table]{xcolor}
\hypersetup{
colorlinks,%
citecolor=black,%
filecolor=black,%
linkcolor=blue,%
urlcolor=black
}
\providecommand{\alert}[1]{\textbf{#1}}

\title{Outline and Tutorial for the OpenBayes Python Module.}
\author{Johannes Castner}
\date{\today}
\hypersetup{
  pdfkeywords={},
  pdfsubject={},
  pdfcreator={Emacs Org-mode version 7.8.09}}

\begin{document}

\maketitle

\setcounter{tocdepth}{3}
\tableofcontents
\vspace*{1cm}


\newpage

\section{Inheritance and Module Organitation}
\label{sec-1}


This is a rough outline of what is contained within the OpenBayes Python module:
\subsection{The \textit{init} file}
\label{sec-1-1}


This file simply imports all of the relevant moduls accessed by users:

\begin{verbatim}
__all__ = ['bayesnet', 'distributions', 'inference',
'potentials', 'table', 'graph', 'OpenBayesXBN', 'BNController']
\end{verbatim}
\subsection{The \textit{bayesnet} file}
\label{sec-1-2}


This file contains three classes: \verb|'BVertex', `BNet'| and \verb|'BNetTestCase'| but it only exports two of them to the end user:

\begin{verbatim}
__all__ = ['BVertex', 'BNet']
\end{verbatim}
\subsubsection{BVertex:}
\label{sec-1-2-1}

This class inherits from \verb|graph.Vertex| and represents one random variable. It is initiated with a name space dictionary containing: \verb|name|, the name of the variable, \verb|discrete| an indicator of whether or not the variable is discrete or continuous, which is either true, meaning discrete (the default) or false, meaning continuous, \verb|nvalues| the number of possible values that the random variable can take on (if it is indeed descrete) and \verb|observed|, which is true if the variable can be observed and false if it is a hidden variable.

\begin{verbatim}
{'name': 'b', 'family': [<bayesnet.BVertex object at 0x2934a10>],
'observed': True, 'discrete': True, 'nvalues': 2, 'distribution': None, '_e': []}
\end{verbatim}
\subsubsection{The methods of BVertex:}
\label{sec-1-2-2}


The methhods of \verb|BVertex| all refer to the \verb|distributions| file, desrcibed further below.
\begin{itemize}
\item \verb|InitDistribution(self, *args, **kwargs)|:
\end{itemize}

initializes conditional distribution of the variable as a function of incoming (causal nodes).

\begin{itemize}
\item \verb|setDistributionParameters(self, *args, **kwargs)|
\end{itemize}

does what its name suggests: it sets the parameters of the distribution.

\begin{itemize}
\item \verb|GetSamplingDistribution(self)|
\end{itemize}

used for the MCMC engine to sample from the distribution.

\begin{itemize}
\item \verb|__cmp__(a,b)|
\end{itemize}

This method is used for correct message passing.
\subsubsection{BNet:}
\label{sec-1-2-3}


This class inherits from \verb|graph.Graph| and it uses the \verb|logging.getLogger(`BNet')| class from the \verb|logging| sub-module. It initializes exactly as \verb|graph.Graph.__init__(self, name)|. Its name-space dictionary looks as:

\begin{verbatim}
{'e': {}, 'name': 'b', 'v': {}}
\end{verbatim}

\end{document}